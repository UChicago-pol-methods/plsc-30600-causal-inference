\documentclass[11pt, article, oneside]{memoir}

%% Required packages
\usepackage{amsmath, amssymb, amsthm}
\usepackage{graphicx, url}
\usepackage{rotating}
\usepackage{multicol}
\usepackage[small,it]{caption}
\usepackage{subfig}
%\usepackage{fullpage, setspace}
\usepackage{epigraph}
\usepackage[all]{xy}
\usepackage{verbatim}
\usepackage[authordate, backend=biber, doi=false, isbn=false,
            backref=true, maxbibnames=10, hyperref=true,
            dateabbrev=false, uniquename=false]{biblatex-chicago}


%% For putting floats at the end
%\usepackage[nolists]{endfloat}


%% XeLaTeX packages + options
\usepackage{mathspec}
\usepackage{xunicode}
\defaultfontfeatures{Mapping=tex-text,Scale=MatchLowercase,Numbers=OldStyle}
%\setsansfont{Helvetica Neue Light}
%\setsansfont{Quicksand}
\setsansfont{Linux Biolinum O}
\setmainfont{Linux Libertine O}
\setmonofont{Inconsolata}
\setmathrm{Linux Libertine O}
\setmathsfont(Latin)[Uppercase=Italic, Lowercase=Italic,
Kerning=Off]{Linux Libertine O}
\setmathsfont(Greek)[Uppercase=Regular, Lowercase=Regular]{Linux Libertine O}
\setmathsfont(Digits)[Numbers={Lining,Proportional}]{Linux Libertine O}


%% Tikz for drawing figures
\usepackage{pgf, tikz}
\usetikzlibrary{positioning}
\usetikzlibrary{arrows}


%% Fancy section labels
\usepackage[compact]{titlesec}
\titleformat{\section}[hang]{\Large\sffamily\bfseries}{\S{\addfontfeatures{Numbers=OldStyle}\thesection}}{1em}{}{}
\titleformat{\subsection}[hang]{\large\sffamily\bfseries}{\addfontfeatures{Numbers=OldStyle}\thesubsection}{1em}{}
\titlespacing*{\section}{0em}{1.5em}{0.5em}
\titlespacing*{\subsection}{0em}{1.5em}{0.5em}


\newcommand{\vs}{\vspace{-\baselineskip}}
\theoremstyle{Assumption}
\newtheorem{assump}{Assumption}
\newcommand{\indep}{\perp\!\!\!\perp}

%% Bibliography files
%\addbibresource{mb.bib}
%\addbibresource{gk.bib}
%\addbibresource{gkpubs.bib}

%% Put url links in titles of bibliography
\ExecuteBibliographyOptions{url=false}
\ExecuteBibliographyOptions{doi=false}
\newbibmacro{string+url}[1]{%
 \iffieldundef{doi}{\iffieldundef{url}{#1}{\href{\thefield{url}}{#1}}}{\href{http://dx.doi.org/\thefield{doi}}{#1}}}
\DeclareFieldFormat{title}{\usebibmacro{string+url}{\mkbibemph{#1}}}
\DeclareFieldFormat[article]{title}{\usebibmacro{string+url}{\mkbibquote{#1}}}
\DeclareFieldFormat[misc]{title}{\usebibmacro{string+url}{\mkbibemph{#1}}}
\DeclareFieldFormat[book]{title}{\usebibmacro{string+url}{\mkbibemph{#1}}}

%% Name, Title, Affiliation, Contact. Change as needed.
\def\myaffiliation{Department of Political Science, University of Chicago}
\def\myauthor{Anton Strezhnev}
\def\myemail{\texttt{\href{mailto:astrezhnev@uchicago.edu}{astrezhnev@uchicago.edu}}}
\def\mywebsite{\mbox{\url{http://www.antonstrezhnev.com}}}
\def\myaddress{Pick Hall 328, 3rd floor, 5828 S University Ave}
\def\mytitle{PLSC 30600: Causal Inference}
\def\mykeywords{Anton Strezhnev, Causal Inference}


%% Custom colors
\definecolor{gray}{rgb}{0.459,0.438,0.471}
\definecolor{crimson}{rgb}{0.34, 0.18, 0.55}


%% Create a command to make a note at the top of the first page describing the
%% publication status of the paper. 
\newcommand{\published}[1]{% 
   \gdef\puB{#1}} 
   \newcommand{\puB}{} 
   \renewcommand{\maketitlehooka}{% 
       \par\noindent\small \puB} 

\usepackage[plainpages=false, 
            pdfpagelabels, 
            bookmarksnumbered,
            pdftitle={\mytitle}, 
            pdfauthor={\myauthor},
            pdfkeywords={\mykeywords},
            colorlinks=true,
            citecolor=crimson, 
            linkcolor=crimson, 
            urlcolor=crimson]{hyperref} 
% \makeatletter
% \newcommand\org@hypertarget{}
% \let\org@hypertarget\hypertarget
% \renewcommand\hypertarget[2]{%
% \Hy@raisedlink{\org@hypertarget{#1}{}}#2%
% } \makeatother


%% blank label items; hanging bibs for text
%% Custom hanging indent for vita items
\def\ind{\hangindent=1 true cm\hangafter=1 \noindent}
\def\labelitemi{$\cdot$}

    % Title flush left
    \pretitle{\begin{flushleft}\Huge\sffamily}
    \posttitle{\end{flushleft}\par\vskip 0.5em}
    \preauthor{\begin{flushleft}\sffamily  \Large \vspace{0.25em}}
    \postauthor{\end{flushleft}}
    \predate{\begin{flushleft}\sffamily \small\vspace{0.9em}}
    \postdate{\end{flushleft}\par\vskip 2em}

\title{{\mytitle}}

\author{\myauthor\smallskip\footnotesize\newline Office: Pick Hall 328, 3rd floor
  \newline Office Hours: Tuesdays 4pm-6pm or schedule an appointment by e-mail \newline
    \myemail \newline \mywebsite
\newline \newline
Teaching Assistant: Zikai Li\newline
Office Hours: Fridays 12:30pm - 1:30pm, Pick Hall, Room 407\newline
\texttt{\href{mailto:zkl@uchicago.edu}{zkl@uchicago.edu}}
}

\published{{\sffamily Winter 2024 | Lecture: Mon./Wed., 3pm - 4:20pm, Lab: Fri., 11:30am - 12:20pm | Room: Stuart Hall 020 (Lecture), Pick Hall 506 (Lab) | Units: 100}}

\counterwithout{section}{chapter}

\date{}
\begin{document}
\maketitle
\textbf{Last Updated: 01/03/2024}
\section*{Course Overview}

Questions of cause and effect are central to the study of political science and to the social sciences more broadly. But making inferences about causation from empirical data is a significant challenge. Critically, there is no simple, assumption-free process for learning about a causal relationship from the data alone. Causal inference requires researchers to make assumptions about the underlying data generating process in order to identify and estimate causal effects. The goal of this course is to provide students with a structured statistical framework for articulating the assumptions behind causal research designs and estimating effects using quantitative data. 

The course begins by introducing the counterfactual framework of causal inference as a way of defining causal quantities of interest such as the ``average treatment effect." It then proceeds to illustrate a variety of different designs for identifying and estimating these quantities. We will start with the most basic experimental designs and progress to more complex experimental and observational methods. For each approach, we will discuss the necessary assumptions that a researcher needs to make about the process that generated the data, how to assess whether these assumptions are reasonable, how to interpret the quantity being estimated and ultimately how to conduct the analysis. 

This course will involve a combination of lectures, sections and problem sets. Lectures will focus on introducing the core theoretical concepts being taught in this course. Sections will emphasize application and demonstrate how to implement various causal inference techniques with real data sets. Problem sets will contain a mixture of both theoretical and applied questions and serve to reinforce key concepts and allow students to assess their progress and understanding throughout the course.

Assignments will involve analysis of data using the R programming language. This is a free and open source language for statistical computing that is used extensively for data analysis in many fields. Prior experience with the fundamentals of R programming is required.

\section*{Prerequisites}

This course is the second in the political science graduate methodology sequence. Completing the introductory course prior to this sequence should prepare you for the material in this class. We will rely on some background knowledge of core concepts in probability, statistics and inference as well as experience with statistical programming in R. However, there are no strict, specific course pre-requisites as many different disciplines and departments offer introductory statistics classes that cover the relevant material. In general, you should have had some introduction to probability theory and should be familiar with concepts like the properties of random variables (especially expectation and variance), estimands and estimators, and statistical inference. Familiarity with regression modeling is a plus but not stricty required. Please contact the instructor at (\href{mailto:astrezhnev@uchicago.edu}{astrezhnev@uchicago.edu}) if you are interested in enrolling but are unsure of the requirements. 

\section*{Logistics}

\textbf{Lectures}: Mondays/Wednesdays from 3pm - 4:20pm -- Location: Stuart Hall, Room 020
\newline\textbf{Sections}: Friday from 11:30am - 12:20pm -- Location: Pick Hall, Room 506
\newline\newline
You should attend sections regularly as they comprise a significant element of the course instruction.
\newline\newline \textbf{Disucssion Forum:} We will use a private \textsc{Ed} discussion forum as a course discussion platform. You should already be enrolled and able to access the forum  See the Canvas page for more details.
\newline\newline\textbf{Course Materials}: Lecture materials, problem sets and section code will be posted on the course GitHub page at \url{https://github.com/UChicago-pol-methods/plsc-30600-causal-inference/}. Problem set solutions will be posted after the due date on Canvas.

Readings will be posted on the Canvas website. You can find them under the ``Modules" section organized by week.


\section*{Textbooks} 

The course will involve readings from a variety of different textbooks and published papers. The class will not require the purchase of a single, specific, text and all excerpts from textbooks are available online (either directly or through library resources). However, we do recommend considering obtaining some of these texts to use as a personal reference and they may be valuable to you in the future.

In general, I have found the following books useful. You do not need to purchase \textit{all} of them, but it is worth being aware of them as they provide very good overviews from a variety of disciplines -- from econometrics to statistics to epidemiology.

\begin{itemize}
\item Cunningham, Scott. \emph{Causal inference: The Mixtape}. Yale University Press, 2021.
\item Huntington-Klein, Nick. \emph{The Effect: An Introduction to Research Design and Causality}. Chapman and Hall/CRC, 2021.
\item Angrist, Joshua D., and Jorn-Steffen Pischke. \emph{Mostly Harmless Econometrics: An Empiricist’s Companion}. Princeton University Press. 2009.
\item Imbens, Guido W. and Donald B. Rubin. \emph{Causal Inference for Statistics, Social, and Biomedical Sciences}. Cambridge University Press.  2010.
\item Hern\'an, Miguel A. and  James M. Robins. \emph{Causal Inference: What If}.  Chapman \& Hall/CRC. 2020. (PDF available at: \url{https://www.hsph.harvard.edu/miguel-hernan/causal-inference-book/})
\item Morgan, Stephen L., and Christopher Winship. \emph{Counterfactuals and Causal Inference.} Cambridge University Press, 2015.
\end{itemize}


\section*{Requirements}


Students’ final grades are based on three components:
\begin{itemize}
\item \textbf{Problem sets} (25\% of the course grade). Students will complete a total of three problem sets throughout the quarter. Problem sets will primarily cover topics from the lecture and section for that week and the previous week.

The goal of the problem sets is to encourage exploration of the material and to provide you with a clear and credible means of assessing your understanding and progress through the course. As such, problem sets are \textit{designed} to be challenging and we expect students to find some questions difficult.

Problem sets will be graded on a (+/\checkmark/-) scale with a + awarded for complete and near-perfect work, a \checkmark awarded for generally good work with clear effort shown but with some errors, and a - awarded for significantly incomplete work with major conceptual errors and little effort shown. 
 
\begin{itemize}
       \item \textit{Collaboration policy}: We strongly encourage collaboration between students on the problem sets and highly recommend that students discuss problems with each other either in person or via Ed. However, each student is expected to submit their own write-up of the answers and any relevant code. 
        \item \textit{Office hours and online discussion}: Students should feel free to discuss any questions about the problem sets with the teaching staff during sections and office hours. We also strongly encourage students to post questions about both the problem sets and the assigned readings on the course \textsc{Ed} discussion board and respond to other students’ questions. Responding to other students’ questions will contribute to your participation grade.
        \item \textit{Submission guidelines}: Problem sets will be distributed as \texttt{PDF} and \texttt{Rmarkdown} files (\texttt{.Rmd}). You should submit your answers and any relevant R code in the same format: including an \texttt{Rmarkdown} file (\texttt{.Rmd} extension) and a corresponding rendered \texttt{.pdf} file as your submission. \texttt{Rmarkdown} combines the text formatting syntax of Markdown markup language with the ability to embed and execute chunks of \texttt{R} code directly into a text document. This allows you to present your code, graphical output, and discussion/write-up all in the same document. We highly recommend that you edit the distributed \texttt{Rmarkdown} assignment file for each problem set directly to make organization easier.
        \end{itemize}
\item  \textbf{Take-home midterm and final exams} (30\% and 35\% of the course grade respectively). The take-home midterm and final will have the same format and structure as the problem sets but with one key difference. You are \textbf{not} permitted to collaborate with other students or any other individual on the exams. The teaching staff will answer any clarifying questions on the \textsc{Ed} discussion board, but will not answer substantive questions.
    \item \textbf{Participation} (10\% of the course grade). We expect students to take an active role in learning in both lecture and section. Engagement with the teaching staff by asking and answering questions will contribute to this grade as will interaction on the \textsc{Ed} board.
\end{itemize}

\section*{Computing}

This course will use the \texttt{R} programming language. This is a free and open source programming language that is available for nearly all computing platforms. You should download and install it from \url{http://www.r-project.org}. Unless you have strong preferences for a specific coding environment, we also highly recommend that you use the free \href{https://rstudio.com}{RStudio} Desktop Integrated Development Environment (IDE) which you can download from \url{https://rstudio.com/products/rstudio/download/#download}. In addition to being a great and simple to use environment for editing code, \texttt{RStudio} makes it very easy to write and compile \texttt{Rmarkdown} documents: the format in which problem sets will be distributed. In addition to base \texttt{R}, we will be frequently using data management and processing tools found in the \href{https://www.tidyverse.org/}{tidyverse} set of packages along with basic graphics and visualization using \href{https://ggplot2.tidyverse.org/}{ggplot2}. 

\subsection*{Policy on Generative Large Language Models}

The rapid growth in both the capabilities and the accessibility of generative large language models (LLMs) such as the GPT series, PaLM, LLaMa, etc... has introduced some novel challenges to the classroom. On the one hand, generative text models can be used as a tool to improve the quality of students' writing. On the other hand, they can be readily used to represent another's work as one's own -- that is, to commit plagiarism. Additionally, LLMs may appear to be useful for some tasks -- such as summarizing a set of texts or finding new sources on a particular topic -- when in fact the outputs are arguably sub-optimal relative to conventional research methods.
\newline\newline
\textbf{My view in short:} Large language models are marvels of \textbf{engineering}. You should use them for \textbf{engineering} tasks, but the task of research is not purely engineering and LLMs are much less effective for the task of doing \textbf{science}.
\newline\newline
By ``engineering," I mean the the iterative task of solving a problem by brainstorming potential solutions, implementing those solutions, and then subsequently \textit{evaluating} the solutions with respect to some clearly defined criteria. The key components here are both the existence of a well-defined problem and the ability to assess whether the proposed solutions are effective.
\newline\newline
Currently, the most obvious and effective use-case for large language models is in coding. I am perfectly happy for you to experiment with using LLMs in debugging code. The interactivity is great for beginning programmers who may have an idea of what they want their code to do, but are unfamiliar with the syntax of a particular language. Likewise, it's an incredibly valuable tool for experienced programmers who want to quickly generate some prototype code that is customized to their particular problem.
\newline\newline
Why is programming an ideal use case? Programming is fundamentally an engineering task. There is a clearly defined problem that a programmer needs to solve via code and there is a straightforward way to evaluate whether a block of code works. As a result, mistakes are easy to catch -- if the code throws an error, something needs to be changed. There is always a human in the loop who is capable of evaluating the output.
\newline\newline
Outside of coding, I do not think LLM outputs are too useful, especially for generating text that is to be submitted without further refinement. In general, you should be cautious about any LLM outputs that you are not able to verify or evaluate yourself. 
\newline\newline
Irrespective of whether LLM outputs are ``good" or not, it is absolutely clear that presenting LLM-generated output as one's own ideas is clearly plagiarism and will be treated as such. This does not rule out all uses of LLM-generated text, but it does rule out most. One use that I would consider acceptable is cleaning up original text that you have written to eliminate grammar mistakes or to rephrase the text to have a clearer style. We already accept the use of spellcheckers and thesauruses that are embedded in most word processors and I don't see this use case as substantively different as long as your original writing is the input. It is important, however, that you are able to evaluate the output and determine that it is conveying exactly what you want to say in exactly the way that you want to say it, just as you would when using any other writing tool.
\newline\newline
Beyond this particular use, \textbf{submitting LLM-generated text as a substitute for your own thinking is not permitted in this class and will be considered plagiarism}. This includes prompting an LLM to compose all or part of your writing and submitting that output either verbatim or with some editing. This policy also applies to generating posts on the Ed discussion board. 
\newline\newline
In general, I do not think that presently there are too many good uses for LLMs for the particular tasks that you will be doing in this class. While these tools have become popular for things like brainstorming, summarizing text, and search, I think that better alternatives exist for the purposes of this class that are more in line with how social scientists conduct research. 

\section*{Schedule}

A schedule of topics and readings is provided below. Each week will cover a single topic or group of topics. Tuesday lectures will typically be an introduction to the topic while Thursday lectures will go into greater detail and involve some applications of the method. You should make sure to review the readings prior to that week's lectures with an aim towards completing the reading assignments prior to Thursday's lecture.

\subsection{Week 1: Introduction to Potential Outcomes (January 3)}

\begin{itemize}
  \item Review of random variables, estimators and inference.
  \item Counterfactual reasoning and the ``Fundamental Problem of Causal Inference"
  \item The ``potential outcomes" model
  \item Estimands and causal quantities of interest
\end{itemize}

\subsubsection*{Readings}

\begin{itemize}
\item Chapter 1, Imbens, Guido W. and Donald B. Rubin. \emph{Causal Inference for Statistics, Social, and Biomedical Sciences}. Cambridge University Press.  2010.
\item Chapter 1, Hern\'an, Miguel A. and  James M. Robins. \emph{Causal Inference: What If}.  Chapman \& Hall/CRC. 2020.
\item Lundberg, Ian, Rebecca Johnson, and Brandon M. Stewart. "What is your estimand? Defining the target quantity connects statistical evidence to theory." \textit{American Sociological Review} 86.3 (2021): 532-565.
\end{itemize}

\textbf{Problem Set 1 Assigned January 3, Due January 16}

\subsection{Week 2: Randomized Experiments (January 8 - January 10)}

\begin{itemize}
\item What assumptions are needed to identify average treatment effects
\item Why randomized experiments satisfy these assumptions
\item Estimation and randomization inference in standard experimental designs
\end{itemize}

\subsubsection*{Readings}

\begin{itemize}
\item Sections 1-5, Athey and Imbens, ``The Econometrics of Randomized Experiments,"  \textit{Handbook of economic field experiments.} Vol. 1. North-Holland, 2017. 73-140. 
\item Chapter 2, Hern\'an, Miguel A. and  James M. Robins. \textit{Causal Inference: What If}.  Chapman \& Hall/CRC. 2020.
\item Druckman, James N., et al. "The growth and development of experimental research in political science." American Political Science Review 100.4 (2006): 627-635.
\end{itemize}

\subsubsection*{Applications}

\begin{itemize}
\item Gerber, Alan S., Donald P. Green, and Christopher W. Larimer. "Social pressure and voter turnout: Evidence from a large-scale field experiment." American political Science review 102.1 (2008): 33-48.
\item Mutz, Diana C., and Byron Reeves. "The new videomalaise: Effects of televised incivility on political trust." American Political Science Review 99.1 (2005): 1-15.
\end{itemize}

\subsection{Week 3: Experiments Continued (January 17)}

\begin{itemize}
\item Stratification and using covariates in experiments
\item Analysis of cluster-randomized experiments
\item Problems of non-compliance
\end{itemize}

\subsubsection*{Readings}

\begin{itemize}
\item Sections 6-12, Athey and Imbens, ``The Econometrics of Randomized Experiments,"  \textit{Handbook of economic field experiments.} Vol. 1. North-Holland, 2017. 73-140. 
\item Montgomery, Jacob M., Brendan Nyhan, and Michelle Torres. "How conditioning on posttreatment variables can ruin your experiment and what to do about it." American Journal of Political Science 62.3 (2018): 760-775.
\item Aronow, P. M., Jonathon Baron, and Lauren Pinson. "A note on dropping experimental subjects who fail a manipulation check." Political Analysis 27.4 (2019): 572-589.
\item Lin, Winston. "Agnostic notes on regression adjustments to experimental data: Reexamining Freedman’s critique." The Annals of Applied Statistics 7.1 (2013): 295-318.
\begin{itemize}
\item \textbf{Bonus:} Samii, C., and P. M. Aronow. "On equivalencies between design-based and regression-based variance estimators for randomized experiments." Statistics \& Probability Letters 82.2 (2012): 365-370.
\end{itemize}

\end{itemize}

\subsubsection*{Applications}

\begin{itemize}
\item Casey, K., Glennerster, R., \& Miguel, E. (2012). Reshaping institutions: Evidence on aid impacts using a preanalysis plan. The Quarterly Journal of Economics, 127(4), 1755-1812.
\item Crépon, B., Devoto, F., Duflo, E., \& Parienté, W. (2015). Estimating the impact of microcredit on those who take it up: Evidence from a randomized experiment in Morocco. American Economic Journal: Applied Economics, 7(1), 123-50.
\end{itemize}


\textbf{Problem Set 2 Assigned January 17, Due January 29}

\subsection{Week 4: Selection-on-observables (January 22 - January 24)}

\begin{itemize}
\item What to do when random assignment of treatment is not possible -- common challenges of observational designs
\item Assumptions behind ``no unobserved confounding" designs
\item Representing assumptions using graphical models
\item Covariate adjustment via subclassification
\end{itemize}

\subsubsection*{Readings}

\begin{itemize}
\item Chapter 12. Imbens and Rubin.
\item Chapters 6-8. Huntington-Klein, Nick. \emph{The Effect: An introduction to research design and causality}. Chapman and Hall/CRC, 2021.
\item Chapter 3 Hern\'an and  Robins
\item Chapter 6-8 Hern\'an and  Robins
\end{itemize}

\subsubsection*{Applications}

\begin{itemize}
\item Washington, Ebonya L. "Female socialization: how daughters affect their legislator fathers." American Economic Review 98, no. 1 (2008): 311-32.
\item Ba, Bocar A., Dean Knox, Jonathan Mummolo, and Roman Rivera. "The role of officer race and gender in police-civilian interactions in Chicago." Science 371, no. 6530 (2021): 696-702.
\end{itemize}

\subsection{Week 5: Selection-on-observables Continued (January 29 - January 31)}

\begin{itemize}
\item Propensity scores and covariate adjustment via weighting 
\item Matching estimators
\item Regression estimators and ``doubly-robust" estimators
\end{itemize}


\subsubsection*{Readings}

\begin{itemize}
\item Chapter 13. Imbens and Rubin. 
\item Imbens, G. W. (2004). Nonparametric estimation of average treatment effects under exogeneity: A review. Review of Economics and statistics, 86(1), 4-29.
\item Aronow, Peter M., and Cyrus Samii. "Does regression produce representative estimates of causal effects?." American Journal of Political Science 60.1 (2016): 250-267.
\item Glynn, Adam N., and Kevin M. Quinn. "An introduction to the augmented inverse propensity weighted estimator." Political Analysis 18.1 (2010): 36-56.
\item Abadie, A., \& Imbens, G. W. (2011). Bias-corrected matching estimators for average treatment effects. Journal of Business \& Economic Statistics, 29(1), 1-11.
\end{itemize}


\textbf{Midterm Exam Assigned January 31, Due February 7}

\subsection{Week 6: Instrumental Variables (February 5 - February 7)}

\begin{itemize}
\item Estimating effects under unobserved confounding using exogenous variation in treatment induced by an instrument
\item Assumptions behind the instrumental variable strategy -- exogeneity, relevance, ``exclusion restriction"
\item Estimation via the Wald Estimator and Two-Stage Least Squares
\item Interpreting the IV estimand -- Local Average Treatment Effect 
\item What makes a good instrument?
\end{itemize}


\subsubsection*{Readings}

\begin{itemize}
\item Cunningham, Causal Inference: The Mixtape, Chapter 7 - Instrumental Variables
\item Angrist, Imbens and Rubin (1996) ``Identification of causal effects using instrumental variables." Journal of the American Statistical Association, 91:434, 444-455
\item Sovey, Allison J., and Donald P. Green. "Instrumental variables estimation in political science: A readers’ guide." American Journal of Political Science 55, no. 1 (2011): 188-200.
\item Andrews, Isaiah, James H. Stock, and Liyang Sun. "Weak instruments in instrumental variables regression: Theory and practice." Annual Review of Economics 11 (2019): 727-753.
\end{itemize}

\subsubsection{Applications}

\begin{itemize}
\item Gerber, Alan S., and Donald P. Green. “The effects of canvassing, telephone calls,
and direct mail on voter turnout: A field experiment.” American political science
review 94.3 (2000): 653-663.
\item Dobbie, Will, Jacob Goldin, and Crystal S. Yang. "The effects of pretrial detention on conviction, future crime, and employment: Evidence from randomly assigned judges." American Economic Review 108.2 (2018): 201-40.
\end{itemize}


\subsection{Week 7: Differences-in-differences (February 12 - February 14)}

\begin{itemize}
\item Weakening ``selection on observables" by studying changes over time. 
\item Assumptions behind the ``differences-in-differences" strategy -- parallel trends
\item Estimation and diagnostics for the identification assumptions.
\item Pitfalls and challenges when units initiate treatment at different times. 
\end{itemize}

\subsubsection*{Readings}

\begin{itemize}
\item Cunningham, The Causal Inference Mixtape, Chapter 9 - Differences-in-differences
\item Roth, J., Sant'Anna, P. H., Bilinski, A., \& Poe, J. (2022). What's Trending in Difference- in-Differences? A Synthesis of the Recent Econometrics Literature. arXiv preprint arXiv:2201.01194.
\item Imai, Kosuke, In Song Kim, and Erik H. Wang. "Matching Methods for Causal Inference with Time‐Series Cross‐Sectional Data." American Journal of Political Science (2021).
\end{itemize}

\subsubsection*{Applications}

\begin{itemize}
\item Malesky, E. J., Nguyen, C. V., \& Tran, A. (2014). The impact of recentralization on public services: A difference-in-differences analysis of the abolition of elected councils in Vietnam. American Political Science Review, 108(1), 144-168.
\item Miller, S., Johnson, N., \& Wherry, L. R. (2021). Medicaid and mortality: new evidence from linked survey and administrative data. The Quarterly Journal of Economics, 136(3), 1783-1829.
\end{itemize}

\textbf{Problem Set 3  Assigned February 12, Due February 26}

\subsection{Week 8: Regression Discontinuity Designs (February 19 - February 21)}

\begin{itemize}
\item Estimating effects under unobserved confounding using quasi-random assignment at a cutpoint.
\item Common applications: Elections, test scores
\item Estimation and sensitivity to modeling assumptions.
\end{itemize}

\subsubsection*{Readings}

\begin{itemize}
\item Chapters 1, 2 and 5. Cattaneo, Matias D., Nicolás Idrobo, and Rocío Titiunik. A practical introduction to regression discontinuity designs: Foundations. Cambridge University Press, 2019.
\item Eggers, A. C., Freier, R., Grembi, V., \& Nannicini, T. (2018). Regression discontinuity designs based on population thresholds: Pitfalls and solutions. American Journal of Political Science, 62(1), 210-229.
\item Keele, Luke J., and Rocio Titiunik. "Geographic boundaries as regression discontinuities." Political Analysis 23.1 (2015): 127-155.
\end{itemize}

\subsubsection*{Applications}

\begin{itemize}
\item Hidalgo, F. Daniel, and Simeon Nichter. "Voter buying: Shaping the electorate through clientelism." American Journal of Political Science 60.2 (2016): 436-455.
\item Bleemer, Zachary, and Aashish Mehta. "Will studying economics make you rich? A regression discontinuity analysis of the returns to college major." American Economic Journal: Applied Economics 14.2 (2022): 1-22.
\end{itemize}

\subsection{Week 9: Mediation and Sensitivity Analysis (February 26 - February 28)}

\begin{itemize}
\item How to define and identify indirect and direct effects of treatment
\item How to assess the robustness of results to violations of identification assumptions. 
\end{itemize}

\subsubsection*{Readings}

\begin{itemize}
\item Blackwell, Matthew. "A selection bias approach to sensitivity analysis for causal effects." Political Analysis 22.2 (2014): 169-182.
\item Cinelli, Carlos, and Chad Hazlett. "Making sense of sensitivity: Extending omitted variable bias." Journal of the Royal Statistical Society: Series B (Statistical Methodology) 82.1 (2020): 39-67.
\item Imai, K., Keele, L., Tingley, D., \& Yamamoto, T. (2011). Unpacking the black box of causality: Learning about causal mechanisms from experimental and observational studies. American Political Science Review, 105(4), 765-789.
\begin{itemize}
\item \textbf{Bonus:} Green, Donald P., Shang E. Ha, and John G. Bullock. "Enough already about “black box” experiments: Studying mediation is more difficult than most scholars suppose." The Annals of the American Academy of Political and Social Science 628.1 (2010): 200-208.
\end{itemize}
\item Acharya, Avidit, Matthew Blackwell, and Maya Sen. "Explaining causal findings without bias: Detecting and assessing direct effects." American Political Science Review 110.3 (2016): 512-529.
\end{itemize}


\textbf{Final Exam Assigned February 28, Due March 7}

\section*{Assignment Schedule}

\begin{itemize}
\item Problem Set 1: Assigned January 3, Due January 16
\item Problem Set 2: Assigned January 17, Due January 29
\item \textbf{Midterm Exam}: Assigned January 31, Due February 7
\item Problem Set 3: Assigned February 12, Due February 26
\item \textbf{Final Exam}: Assigned February 28, Due March 7 
\end{itemize}

\section*{Acknowledgments}

This course is indebted to the many wonderful and generous scholars who have developed causal inference curricula in political science departments throughout the world and who have made their course materials available to the public. In particular, I thank Matthew Blackwell, Brandon Stewart, Molly Roberts, Kosuke Imai, Teppei Yamamoto, Jens Hainmueller, Adam Glynn, Gary King, Justin Grimmer whose lecture notes and syllabi have been immensely valuable in the creation of this course. I also thank Andy Eggers, Molly Offer-Westort and Bobby Gulotty whose comments and feedback have been essential to the development of this class.

Special thanks to previous TAs of this class: Arthur Yu and Oscar Cuadros


\end{document}
 
